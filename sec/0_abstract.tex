
\begin{abstract}

With the rise of AI-generated sensitive images, knowing the source of content is critical to distinguish AI generated images. Traditional image watermarking methods lack robustness to common image transformations such as filters, lossy compression during social-media sharing and screenshots. Watermarks can also be faked or removed if models are open-sourced or leaked, since images can be rewatermarked. We have developed a three-part framework for secure, transformation-resilient AI content provenance detection, to address these limitations. We develop an adversarially robust state-of-the-art perceptual hashing model, DinoHash, derived from DINOV2, robust to common transformations like filters, compression and crops. Additionally, we integrate a Multi-Party Fully Homomorphic Encryption~(MP-FHE) scheme into our proposed framework to ensure the protection of both user queries and registry privacy. Furthermore, we improve previous work on AI-generated media detection. This approach is useful in cases where the content is absent from our registry. DinoHash achieves a significant improvement in average bit accuracy, showing a $12\%$ increase over state-of-the-art watermarking and perceptual hashing methods, with a consistent edge in true positive rate (TPR) and false positive rate (FPR) tradeoffs across a wide range of transformations. Our AI-generated media detection results perform consistently better showing a $25\%$ improvement in classification accuracy on commonly used real-world AI image generators than existing algorithms. By combining perceptual hashing, MP-FHE, and an AI content detection model, our proposed framework provides better robustness and privacy compared to previous work.
% \clearpage


% In a perfect world, you might imagine that all content on the web is marked with some kind of invisible identifier. The identifier could contain a cryptographic signature attesting to the origin of the content, or additional metadata like the model used to create a deepfake or the approximate location of a photo. The ideal identifier should be imperceptible to human senses and impervious to algorithmic analysis. It should also remain secure against both unintentional attacks like compression, screenshotting, and re-capture, and intentional attacks, such as histogram or Gaussian noise manipulation. Unfortunately, watermarking schemes do not have proper robustness against “realistic” internet interactions such as JPEG compression, a non-corrupting crop, or screenshots. \\

% Embedding-derived perceptual hashes allow us to track a content’s perceptual representation in a quick and robust way, giving us a “weak” mapping from an image to similar images. In addition, they are more robust against aforementioned attacks, compared to watermarking. Finally, they can be compared in a completely privacy preserving manner. \\

% We introduce a state-of-the-art perceptual hash (p-hash) algorithm for image similarity, as well as a novel system for indexing images using p-hash values and verifying content originality through MPC queries. To ensure privacy, the p-hash values are stored encrypted in an external database. Authorities such as content providers can use a shared public key to query whether query images are already in the database, or are unique.

\end{abstract}
